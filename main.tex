\documentclass{moderncv}
\moderncvstyle{classic}

\usepackage[scale=0.86]{geometry}

\name{Joe}{Down}
%\address{Flat 2, 406C, Hackney Road}{E2 7AP, London}{}
\title{Computer Science Graduate}
\email{joe-down@outlook.com}
\phone[mobile]{+44~7427~251~937}
\social[github]{joe-down}
\social[linkedin]{joe-down}

\begin{document}
    \makecvtitle

    \section{Experience}
        \cventry{Jun~2023--Aug~2023}{Software Engineering Intern}{Sky}{London}{}{
            \begin{itemize}
                \item Used PyTorch to train a tennis live-stream event detector based on `Multi-modal Self-Supervision from Generalized Data Transformations'.
                \item Sorted and extracted appropriate training data from available match footage (serve, rally, foul, etc.).
                \item Integrated with score detector from previous internship to produce a unified detector with additional optimisations and validation.
            \end{itemize}
        }
        \cventry{Jun~2022--Sep~2022}{Software Engineering Intern}{Sky}{London}{}{
            \begin{itemize}
                \item Developed automated score detectors for Premier League football and tennis matches using computer vision (OpenCV) and optical character recognition Python libraries.
                \item Implemented appropriate unit tests.
                \item Dockerised to run on an AWS EC2 instance, taking a live video stream as input and emitting appropriate event notifications to Amazon SNS.
            \end{itemize}
        }
        \cventry{2019--2020}{Tutor}{Explore Learning}{High Wycombe}{}{
            \begin{itemize}
                \item Group and one-to-one English and Maths tutoring for students ages 4-16.
                \item Taught small group 11+ exam preparation classes.
            \end{itemize}
        }

    \section{Education}
        \cventry{2020--2024}{MEng Computer Science}{University College London}{}{First Class Honours}{
        %\textbf{Year 1:} Principles of Programming, Theory of Computation, Object-Oriented Programming, Algorithms, Compilers, Discrete Mathematics for Computer Scientists, Engineering Challenges, Design and Professional Skills 1
        %\\\textbf{Year 2:} Computer Architecture and Concurrency, Logic and Database Theory, Software Engineering, Mathematics and Statistics, Intelligent Systems, Systems Engineering, Security
        %\\\textbf{Year 3:} Computability and Complexity Theory, Computer Systems, Artificial Intelligence and Neural Computing, Computer Graphics, Research Methods, Group Research Project, Machine Learning and Neural Computing, Robotic Systems
        %\\\textbf{Year 4:} Supervised Learning, Applied Machine Learning, Virtual Environments, Multi-agent Artificial Intelligence, Robotic Systems Engineering, Machine Vision, Individual Project
        \begin{itemize}
            \item\textbf{Optional Modules:} Intelligent Systems, Artificial Intelligence and Neural Computing, Computer Graphics, Machine Learning and Neural Computing, Robotic Systems, Supervised Learning, Applied Machine Learning, Virtual Environments, Multi-agent Artificial Intelligence, Robotic Systems Engineering, Machine Vision
            %\begin{itemize}
                %\item\textbf{Year 2:} Intelligent Systems
                %\item\textbf{Year 3:} Artificial Intelligence and Neural Computing, Computer Graphics, Machine Learning and Neural Computing, Robotic Systems
                %\item\textbf{Year 4:} Supervised Learning, Applied Machine Learning, Virtual Environments, Multi-agent Artificial Intelligence, Robotic Systems Engineering, Machine Vision
            %\end{itemize}
            \item\textbf{Key Projects:}
            \begin{itemize}
                \item\textbf{Year 3 Group Research Project:}
                \textit{Parameter-Wise Double Descent - A Unified Model or Not?}
                \begin{itemize}
                    \item Investigated and questioned the ubiquity of the double descent phenomenon by training a range of neural networks on the MNIST dataset.
                \end{itemize}
                \item\textbf{Year 4 Individual Project:}
                \textit{WebMGA 3.0.}
                \begin{itemize}
                    \item Web tool for interactive 3D visualisation of molecular configurations for liquid crystals.
                    \item Developed using Node.js, React, and three.js.
                \end{itemize}
                \item\textbf{Virtual Environments Group Project:}
                \begin{itemize}
                    \item A tool for teaching and learning sign language using hand tracking in a networked VR environment.
                    \item Developed using Unity for Meta Quest 2 and 3 headsets.
                \end{itemize}
                \item\textbf{Multi-agent Artificial Intelligence Group Project:}
                \textit{Identifying how agents behave in the `Warlords' game environment when trained using different multi-agent reinforcement learning techniques. Can interesting behaviours occur or be promoted?}
                \begin{itemize}
                    \item Implemented `Deep Q-Learning' and `Multi Agent Deep Deterministic Policy Gradient' algorithms to try to learn behaviours in the 4-player Atari game `Warlords'.
                \end{itemize}
            \end{itemize}
        \end{itemize}
        }
        %\cventry{2012--2019}{Secondary School}{The Royal Grammar School}{High Wycombe}{}{}

    %\section{Projects}

    \section{Skills}
        \cvitem{Programming}{Python, JavaScript/TypeScript, C, C++, C\#, Java, Bash, SQL, Haskell, WebGL, TeX}
        \cvitem{Frameworks}{PyTorch, TensorFlow, Docker, Unity, Node.js, React, PyQt, Flask, Rospy, OpenCV}
        \cvitem{Tools}{Git, AWS (ECS, EC2, S3, SageMaker), Jupyter, JetBrains IDEs}
        \cvitem{Systems}{Linux (Arch, Nix, Debian), Arduino, Raspberry Pi, Windows, Meta Quest}
        \cvitem{Other}{Professional Scrum Master I}

\end{document}
